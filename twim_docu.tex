\documentclass{report}
\usepackage{geometry}
 \geometry{
 a4paper,
 total={170mm,257mm},
 left=20mm,
 top=20mm,
 }
\usepackage[latin1]{inputenc}% erm\"oglich die direkte Eingabe der Umlaute 
\usepackage[T1]{fontenc} % das Trennen der Umlaute
\usepackage{ngerman} % hiermit werden deutsche Bezeichnungen genutzt und 
                     % die W\"orter werden anhand der neue Rechtschreibung 
             % automatisch getrennt. 
\usepackage{graphicx}
\usepackage{comment}
\usepackage{hyperref}
\graphicspath{ {./} }
\usepackage{subcaption}
\newcommand{\sectionbreak}{\clearpage}
\usepackage[section]{placeins}
\usepackage{subcaption}
\usepackage{mathtools}
\title{Step by step TWIM Calibration}
\begin{document}

\maketitle

\section{Overview}
For this calibration I used the data of Experiment S455 in March 2021, Run 273, subruns 1-48.
\begin{enumerate}
	\item Alignment of the energy per Anode for each Section
	\item Alignment of the energy per Section
\end{enumerate}
Info: I always count from 0 to 15.

\section{Alignment of the energy per Anode for each Section}
For the Energy, you should first align all the gain per anode by plotting for each section:\newline 
Eraw[anode i] vs Eraw[anode ref].\newline
anode ref = the 5th anode\newline
I plot these 2D histos only for events where the 16 anodes per section have seen an ion. (no specific tpat selection needed).
\subsection{Computing}
First run the program \dq small\textunderscore script\textunderscore hist.C\dq{} for all subruns. Then use \dq hadd\dq{} to add up the .root
files. The combined .root file can then be used for the scrip called \dq retrieve\textunderscore fits\textunderscore hist.C \dq{}. This one makes nice canvases for the plots anode[i] vs anode\textunderscore ref and stores the fit parameters under 
parameters\textunderscore twim\textunderscore anodes.csv.\\
In this directory you find the parameters\textunderscore twim\textunderscore anodes.csv I retrieved from this first calibration step. The offset (=gain) should be near to 1.\newline
\subsubsection{Design of parameters\textunderscore twim\textunderscore anodes.csv}
It stores section(s), anodenr(i),slope,offset\newline
$E\textunderscore sum\textunderscore ref*slope[i][s] + offset[i][s]$\newline
To calibrate you have to do it the other way round:\newline
$E\textunderscore cal\textunderscore anode[i][s] = E\textunderscore anode[i][s]/slope[i][s] -offset[i][s]/slope[i][s]$
\section{Alignment of the energy per Section}
\begin{itemize}
	\item you should select event where the ions loss their energy in one section only
	\item then you should select a limited range in ToF (100 ps range)
	\item then you should calculate for each section Esum[s], the sum of the 16 anodes ( using $E\textunderscore cal\textunderscore anode[i][s]$ from the previous step). 
	\item you should see a small shift between the four sections
	\item correct from this shift by pol1
	\item as result you get: $Eal\textunderscore step2final[s][a] = OffsetPerSection[s] + GainPerSection[s] * E\textunderscore cal\textunderscore anode[i][s]$
\end{itemize}
\subsection{Computing}
Run the macro \dq twim\textunderscore sum\textunderscore energy.C\dq{} using all subruns as input parameter. As output you get a .root file with 1D histos with the summed TWIM energy for each section. Use this output .root file as input file for the macro \dq e\textunderscore sum\textunderscore cal.C\dq{}.This uses TSpectra etc. to (linearly) calibrate the E\textunderscore sum energy for all sections. The according fit parameters are stored in the parameter file \dq sum\textunderscore anodes\textunderscore parameters.csv\dq{}.\newline
Now you can use the macro \dq twim\textunderscore final\textunderscore cal.C\dq{} (as input parameter the name of the subruns). This macro uses both parameter files \dq parameters\textunderscore twim\textunderscore anodes.csv (anode fit) and \dq sum\textunderscore anodes\textunderscore parameters.csv\dq{}(summed energy fit). Now E\textunderscore sum is calibrated.



\section{Preparation of the MWPCs}




\end{document}

