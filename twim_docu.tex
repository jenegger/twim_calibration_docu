\documentclass{report}
\usepackage{geometry}
 \geometry{
 a4paper,
 total={170mm,257mm},
 left=20mm,
 top=20mm,
 }
\usepackage[latin1]{inputenc}% erm\"oglich die direkte Eingabe der Umlaute 
\usepackage[T1]{fontenc} % das Trennen der Umlaute
\usepackage{ngerman} % hiermit werden deutsche Bezeichnungen genutzt und 
                     % die W\"orter werden anhand der neue Rechtschreibung 
             % automatisch getrennt. 
\usepackage{graphicx}
\usepackage{comment}
\usepackage{hyperref}
\graphicspath{ {./} }
\usepackage{subcaption}
\newcommand{\sectionbreak}{\clearpage}
\usepackage[section]{placeins}
\usepackage{subcaption}
\title{Step by step TWIM Calibration}
\begin{document}

\maketitle

\section{Overview}
For this calibration I used the data of Experiment S455 in March 2021, Run 273, subruns 1-48.
\begin{enumerate}
	\item Alignment of the energy per Anode for each Section
	\item Alignment of the energy per Section
\end{enumerate}
Info: I always count from 0 to 15.

\section{Alignment of the energy per Anode for each Section}
For the Energy, you should first align all teh gain per anode by plotting for each section:\newline 
Eraw[anode i] vs Eraw[anode ref].\newline
anode ref = the 5th anode\newline
I plot these 2D histos only for events where the 16 anodes per section have seen an ion. (no specific tpat selection needed).
\subsection{Computing}
First run the program \dq small\textunderscore script\textunderscore hist.C\dq{} for all subruns. Then use \dq hadd\dq{} to add up the .root
files. The combined .root file can then be used for the scrip called \dq retrieve\textunderscore fits\textunderscore hist.C \dq{}. This one makes nice canvases for the plots anode[i] vs anode\textunderscore ref and stores the fit parameters under 
parameters\textunderscore twim\textunderscore anodes.csv.



\section{Calibration of the sections}

\section{Preparation of the MWPCs}




\end{document}

